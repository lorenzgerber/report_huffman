\documentclass[a4paper,11pt,twoside]{article}
%\documentclass[a4paper,11pt,twoside,se]{article}

\usepackage{UmUStudentReport}
\usepackage{verbatim}   % Multi-line comments using \begin{comment}
\usepackage{courier}    % Nicer fonts are used. (not necessary)
\usepackage{pslatex}    % Also nicer fonts. (not necessary)
\usepackage[pdftex]{graphicx}   % allows including pdf figures
\usepackage{listings}
%\usepackage{lmodern}   % Optional fonts. (not necessary)
%\usepackage{tabularx}
%\usepackage{microtype} % Provides some typographic improvements over default settings
%\usepackage{placeins}  % For aligning images with \FloatBarrier
%\usepackage{booktabs}  % For nice-looking tables
%\usepackage{titlesec}  % More granular control of sections.

% DOCUMENT INFO
% =============
\department{Institution för Datavetenskap}
\coursename{Datavetenskapens byggstenar 7.5 p}
\coursecode{DV160HT15}
\title{OU3 Huffman Coding}
\author{Simon Andersson ({\tt{dv15san@cs.umu.se}})}
\author{Lorenz Gerber ({\tt{dv15lgr@cs.umu.se}})} 
\date{2016-02-18}
%\revisiondate{2016-01-18}
\instructor{Lena Kallin Westin / Erik Moström / Lina Ögren}


% DOCUMENT SETTINGS
% =================
\bibliographystyle{plain}
%\bibliographystyle{ieee}
\pagestyle{fancy}
\raggedbottom
\setcounter{secnumdepth}{2}
\setcounter{tocdepth}{2}
%\graphicspath{{images/}}   %Path for images

\usepackage{float}
\floatstyle{ruled}
\newfloat{listing}{thp}{lop}
\floatname{listing}{Listing}


% DEFINES
% =======
%\newcommand{\mycommand}{<latex code>}

% DOCUMENT
% ========
\begin{document}
\lstset{language=C}
\maketitle
\thispagestyle{empty}

\tableofcontents
\newpage

\clearpage
\pagenumbering{arabic} 

\section{Introduction} 
The aim of this laboration was to plan and implement a command line
program written in C that accomplishes encoding and decoding of 
text files according to the \emph{Huffman} algorithm.

The \emph{Huffman} algoritm is used for data compression, in our case
for text files. The basic idea is that instead of using 8 bytes for
every character, fine unique variable length binary representations
for every character where the most common used characters get the
shortest binary sequences. In practice, this includes several steps:
First a frequency count table of all characters has to be
compiled. Then a binary tree is constructed where characters and their
count frequency as leafs. Characters with high frequency count will be
placed the closest to the tree root. A more detailed description of
this process will be given in the method description.



\section{Material and Methods}
\subsection{Datatypes}
From the provided datatypes we used \emph{prioqueue} (which is built on
\emph{list\_2cell}), \emph{tree\_3cell}, and \emph{bitset}. 



\subsection{Work Organization}
On an initial kick-off meeting, we discussed the problem and possible
solution strategies. Then we created repositories for the 
\emph{\href{https://github.com/lorenzgerber/huffman}{code}} and for
the \emph{\href{https://github.com/lorenzgerber/report_huffman}{report}}
on \emph{\href{https://github.com}{github}} and setup a team in a 
workgroup messaging  app (\emph{\href{https://slack.com}{slack}}). All
further work and communication was done remote using the
afore-mentioned tools. 

\section{Results}
And our results looked like this...

\section{Discussion}
bla bla bla...

\section{Contributions}
Both authors were involved in every function with either writing or
debugging/checking it. The initials in table \ref{tab:contribution}
stand for the person who initially wrote the respective function.


\begin{table}[]
\centering
\caption{work contributionsn}
\label{tab:contribution}
\begin{tabular}{ll}
planning and defining the strategy & SAN, LGR \\
setting up and managing git repos  & LGR      \\
                                   &          \\
initial code structure             & SAN, LGR \\
main program and argument handling & LGR      \\
char frequency count               & SAN      \\
compare tree function              & SAN      \\
build Huffman tree                 & SAN      \\
tree traversal function            & SAN      \\
huffman code from tree traversal   & LGR      \\
encode function                    & LGR      \\
decode function                    & SAN, LGR \\
file read/write                    & LGR      \\
                                   &          \\
commenting and styling code        & SAN, LGR \\
memory leak check                  & SAN      \\
                                   &          \\
setting up LaTex document          & LGR      \\
writing report                     & SAN LGR  \\
                                   &         
\end{tabular}
\end{table}


\addcontentsline{toc}{section}{\refname}
\bibliography{references}

\end{document}
